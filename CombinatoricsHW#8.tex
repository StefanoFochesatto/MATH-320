%%% Preamble starts here.
\documentclass{amsart}
\usepackage{stmaryrd}
%for the heading
\usepackage{fancyhdr, enumerate}
%for the picture. 
\usepackage{tikz}
%adjust the page width
\usepackage[margin=1in]{geometry}
\usepackage{mathtools}
\def\multiset#1#2{\ensuremath{\left(\kern-.3em\left(\genfrac{}{}{0pt}{}{#1}{#2}\right)\kern-.3em\right)}}
\linespread{1.1}

\linespread{1.1}

%command for double parentheses
\newcommand{\textmultiset}[2]{\bigl(\!{\binom{#1}{#2}}\!\bigr)}
\newcommand{\displaymultiset}[2]{\left(\!{\binom{#1}{#2}}\!\right)}
\newcommand\multiset[2]{\mathchoice{\displaymultiset{#1}{#2}}
                                {\textmultiset{#1}{#2}}
                                {\textmultiset{#1}{#2}}
                                {\textmultiset{#1}{#2}}}

%special commands for number sets
\def\RR{{\mathbb R}}
\def\NN{{\mathbb N}}
\def\ZZ{{\mathbb Z}}
\def\QQ{{\mathbb Q}}
\def\CC{{\mathbb C}}

% header
\lhead{\sc  Combinatorics: Homework 8}
\chead{\sc Stefano Fochestto } 
\rhead{\today}
\cfoot{}
\pagestyle{fancy}

%%%% Main document starts here.

\begin{document}
\thispagestyle{fancy}

 
\begin{enumerate}
%%%first problem
\item (Problem 3.4.2) Derive a combinatorial identity via the equation
$$\frac{1}{(1-x)^{m+n}}=\frac{1}{(1-x)^m}\cdot\frac{1}{(1-x)^n}.$$ (You should justify your answer.)

\textbf{Answer:}
Consider the equation is an equality of two ordinary generating functions.
\begin{equation*}
\big\llbracket\frac{1}{(1-x)^{m+n}}\big\rrbracket=\big\llbracket\frac{1}{(1-x)^m}\cdot\frac{1}{(1-x)^n}\big\rrbracket
\end{equation*}
From here along with the convolution formula for ordinary generating function we get,
\begin{equation*}
\multiset{n+m}{k} = \sum_{j=0}^{k} \multiset{m}{j} \multiset{n}{k-j}
\end{equation*}
\begin{equation*}
{n+m+k-1 \choose k} = \sum_{j=0}^{k} {m+j-1 \choose j} {n+k-j-1 \choose k-j}
\end{equation*}

I suppose that it's some variation of Vandermonde's Identity but with multi-sets.


\vspace{1in}

\item (Problem 3.4.5) Let $a$, $b$, and $c$ be nonzero real numbers. Find the coefficient of $x^k$ in $\displaystyle{\frac{a}{b+cx}}.$\\
\textbf{Answer:} Through a little bit of algebra we can simplify this problem,
\begin{align*}
\frac{a}{b+cx} &= a \frac{1}{b+cx}\\
& = \frac{a}{b} \frac{1}{1+(\frac{c}{b})x}
\end{align*}
Because of the linearity of coefficient extraction we can see that the coefficient of $x^k$ will be $\frac{a}{b}*(-\frac{c}{b})^k$
\vspace{1in}

\item (Problem 3.4.9) Suppose the EGF of $\{c_n\}_{n\geq 0}$ is $(e^x-1)^2.$ Find a formula for $c_n.$\\

\textbf{Answer:} Again we can just do a little bit of algebra to simplify the EGF,
\begin{align*}
(e^x-1)^2 &= (e^x - 1)(e^x - 1)\\
&= e^{2x}-2e^x+1
\end{align*}
Then from here we split up the EGF and solve. We get $c_n = 2^n - 2$. 
\vspace{1in}

\item (Problem 3.4.11) Here is how Euler proved that the binary representation of any nonnegative integer is unique. For $n \geq 0,$ let $b_n$ denote the number of ways to write $n$ as a sum of powers of 2. Let $B(x)$ be the OGF of $\{b_n\}_{n \geq 0}.$\\
	\begin{enumerate}
	\item Explain why $B(x)=(1+x)(1+x^2)(1+x^4)(1+x^8)(1+x^{16})\cdots.$\\
	\textbf{Answer:} Consider that $n = z_0+z_1+z_2+..$ and that for each $z_i \in \{0,2^i\}$ If we wanted to find the number of solutions to $n$ with the given constraint on $z_i$ the corresponding generating function would be, $B(x)$.
	\vspace{.5in}
	
	\item Explain why $B(x)=(1+x)B(x^2).$
	
	\textbf{Answer:} We NTS that the LHS can simplify to
	\begin{equation*}
	B(x) = \displaystyle{\prod_{i=0}^\infty (1+x^{2^i})}
	\end{equation*}
	We can do this through some algebra,
	\begin{align*}
	(1+x)B(x^2) &= (1+x)(1+x^{2*1})(1+x^{2*4})(1+x^{2*8})...\\
	&= (1+x)(1+x^{2})(1+x^{8})(1+x^{16})...\\
	&=\displaystyle{\prod_{i=0}^\infty (1+x^{2^i})} 
	\end{align*}

	\vspace{1in}
	
	\item Use part (b) to prove that $b_n=1$ for all $n \geq 0.$\\
	
	\textbf{Answer:} So first of all lets expand the first two terms,
	\begin{align*}
	B(n) &= (1+x)(1+x^2)\displaystyle{\prod_{i=2}^\infty (1+x^{2^i})} \\
	&= (1+x+x^2+x^3)\displaystyle{\prod_{i=2}^\infty (1+x^{2^i})}\\
	&= (1+x+x^2+x^3)(1+x^4)\displaystyle{\prod_{i=3}^\infty (1+x^{2^i})} \text{ Moving the third term out}\\
	&= (1+x+x^2+x^3+x^4+x^5+x^6+x^7)\displaystyle{\prod_{i=3}^\infty (1+x^{2^i})}
	\end{align*}
	As you can see if we continue the infinite product simplifies to a geometric series,
	\begin{equation*}
	\sum_{n=0}^{\infty} x^k
	\end{equation*}
	and we know the coefficient on each term will always be 1. Thus  $b_n=1$ for all $n \geq 0.$
	\vspace{1in}
	\end{enumerate}



\item (Problem 3.5.1) Solve the following recurrence relations using the generating function technique.\\

	 (a) $a_0=0$ and $a_n=2a_{n-1}+1$ for $n\geq 1$\\
	 
	 \textbf{Answer:} First suppose that $f(x) = \sum_{n = 0}^{\infty} a_n x^n$ is the OGF that describes the sequence $\{a_n\}_{n\geq0}$. Now through some algebra on the recurrence we can get a concise form for our OGF,
	 \begin{align*}
	 a_n&=2a_{n-1}+1\\
	 a_nx^n&=2a_{n-1}x^n+x^n \text{   Multiply through by $x^n$}\\
	 \sum_{n = 1}^{\infty} a_nx^n&=\sum_{n = 1}^{\infty}2a_{n-1}x^n+\sum_{n = 1}^{\infty}x^n \text{ Summing over values which the recurrence is defined}\\
	 \sum_{n = 0}^{\infty} a_nx^n - a_0 &=2x\sum_{n = 1}^{\infty}a_{n-1}x^{n-1}+\sum_{n = 1}^{\infty}x^n \text {  Simplifying each sum}\\
	 f(x) &= 2xf(x)+\frac{x}{1-x} \text{   Substituting concise OGF}\\
	  f(x) &=\frac{x}{(1-x)(1-2x)}\text{   Solving for $f(x)$}\\
	  f(x) &= \frac{-x}{(1-x)}+\frac{2x}{1-2x} \text{  Partial Fraction}\\
	  a_n &= 2^n+1
	 \end{align*}
	 Thus the closed form for the given recurrence relation is $a_n = 2^n+1$.
	\vspace{.5 in}
	 
	 (c) $c_0=1$ and $c_n=3c_{n-1}+3^n$ for $n\geq 1$\\
	 
	 \textbf{Answer:} First suppose that $f(x) = \sum_{n = 0}^{\infty} c_n x^n$ is the OGF that describes the sequence $\{c_n\}_{n\geq0}$. Now through some algebra on the recurrence we can get a concise form for our OGF,
	  \begin{align*}
	 c_n&=3c_{n-1}+3^n\\
	 c_nx^n&=3c_{n-1}x^n+3^nx^n \text{   Multiply through by $x^n$}\\
	 \sum_{n = 1}^{\infty} c_nx^n&= \sum_{n = 1}^{\infty}3c_{n-1}x^n+ \sum_{n = 1}^{\infty}3^nx^n \text{ Summing over values which the recurrence is defined}\\
	\sum_{n = 0}^{\infty} c_nx^n - c_0 &= 3x\sum_{n = 1}^{\infty}c_{n-1}x^{n-1}+ \sum_{n = 1}^{\infty}3^n x^n \text {  Simplifying each sum} \\
	f(x) -1 &= 3xf(x)+ \frac{1}{(1-3x)} - 1\text{   Substituting concise OGF} \\
	f(x) &= \frac{3}{(1-3x)^2} \text{   Solving for $f(x)$}\\
	c_n &= 3^n \multiset{2}{n}\\
	c_n &= (n+1)3^n \text{ Simplifying multiset}
	 \end{align*}
		 Thus the closed form for the given recurrence relation is $c_n = (n+1)3^n$.
	\vspace{.5 in}
	 
	 (e) $e_0=e_1=1,\:e_2=2$ and $e_n=3e_{n-1}-3e_{n-2}+e_{n-3}$ for $n\geq 3$\\
	
	\textbf{Answer:} First suppose that $f(x) = \sum_{n = 0}^{\infty} e_n x^n$ is the OGF that describes the sequence $\{e_n\}_{n\geq0}$. Now through some algebra on the recurrence we can get a concise form for our OGF,
	 \begin{align*}
	 e_n&=3e_{n-1}-3e_{n-2}+e_{n-3}\\
	 e_nx^n&=3e_{n-1}x^n-3e_{n-2}x^n+e_{n-3}x^n \text{   Multiply through by $x^n$}\\
	 \sum_{n = 3}^{\infty} e_nx^n&=\sum_{n = 3}^{\infty} 3e_{n-1}x^n- \sum_{n = 3}^{\infty} 3e_{n-2}x^n+ \sum_{n = 3}^{\infty}e_{n-3}x^n  \text{ Summing the recurrence}\\
 	 \sum_{n = 0}^{\infty} e_nx^n - (e_0+e_1+e_2)&=3x\sum_{n = 0}^{\infty}e_{n-1}x^{n-1}-(e_0 + e_1) - 3x^2\sum_{n = 0}^{\infty} e_{n-2}x^{n-2}-(e_0)+ x^3\sum_{n = 3}^{\infty}e_{n-3}x^{n-3} \\
	  f(x) - (e_0+e_1+e_2)&=3x(f(x)-(e_0 + e_1)) - 3x^2(f(x)-e_0)+ x^3f(x) \text{   Substituting concise OGF}\\
	  f(x) &= \frac{-3x^2}{(x-1)^3}+\frac{6x}{(x-1)^3}-\frac{4}{(x-1)^3} \text{   Solving for $f(x)$}\\
	  e_n &= -3\multiset{3}{n-2}+6\multiset{3}{n-1}-4\multiset{3}{n}
	 \end{align*}
	 Thus the closed form for the given recurrence relation is $  e_n = -3\multiset{3}{n-2}+6\multiset{3}{n-1}-4\multiset{3}{n}$.

	\vspace{.5 in}

\item (Problem 3.5.2) Use an EGF to solve the recurrence relation $a_0=2$ and $a_n=na_{n-1}-n!$ for $n \geq 1.$\\
	\textbf{Answer:} First suppose that $f(x) = \sum_{n = 0}^{\infty} a_n \frac{x^n}{n!}$ is the EGF that describes the sequence $\{a_n\}_{n\geq0}$. Now through some algebra on the recurrence we can get a concise form for our EGF,\\
	 \begin{align*}
	 a_n&=na_{n-1}-n! \\
	 a_n\frac{x^n}{n!}&=na_{n-1}\frac{x^n}{n!}-n!\frac{x^n}{n!}  \text{   Multiply through by $\frac{x^n}{n!}$}\\
	 \sum_{n = 1}^{\infty} a_n\frac{x^n}{n!}&=\sum_{n = 1}^{\infty}na_{n-1}\frac{x^n}{n!}-\sum_{n = 1}^{\infty}n!\frac{x^n}{n!}   \text{ Summing over recurrence} \\
	 \sum_{n = 0}^{\infty} a_n\frac{x^n}{n!} - a_0 &=x\sum_{n = 1}^{\infty}a_{n-1}\frac{x^{n-1}}{(n-1)!}-\sum_{n = 1}^{\infty}n!\frac{x^n}{n!}   \text{ Simplifying the sum} \\
	  f(x) - 2 &=xf(x)-(\frac{1}{1-x} - 1) \text{   Substituting concise EGF} \\
	  f(x) &= \frac{3}{1-x}-\frac{1}{(1-x)^2} \text{   Solving for $f(x)$}\\
	  a_n &= 3n! - n!\multiset{2}{n}\\
	   a_n &= (2-n)n!  \text{   Simplifying multiset}
	 \end{align*}
	  Thus the closed form for the given recurrence relation is $a_n =(2-n)n!$
	\vspace{.5 in}


\item (Problem 3.5.4) Find a formula for the $n$th term of the sequence defined by the recurrence relation $E_n=nE_{n-1}+(-1)^n$ for $n \geq 1$ and $E_0=1.$ Also, what is the relationship between $E_n$ and $D_n,$ the number of derangements of $[n]$?\\

	\textbf{Answer:} First suppose that $f(x) = \sum_{n = 0}^{\infty} E_n \frac{x^n}{n!}$ is the EGF that describes the sequence $\{E_n\}_{n\geq0}$. Now through some algebra on the recurrence we can get a concise form for our EGF,\\
	 \begin{align*}
	 E_n&=nE_{n-1}+(-1)^n\\
	 E_n\frac{x^n}{n!}&=nE_{n-1}\frac{x^n}{n!}+(-1)^n\frac{x^n}{n!} \text{  Multiply through by $\frac{x^n}{n!}$}\\
	 \sum_{n = 1}^{\infty} E_n\frac{x^n}{n!}&= \sum_{n = 1}^{\infty} nE_{n-1}\frac{x^n}{n!}+ \sum_{n = 1}^{\infty}(-1)^n\frac{x^n}{n!} \text{ Summing over recurrence}\\
	 \sum_{n = 0}^{\infty} E_n\frac{x^n}{n!} - E_0 &= x \sum_{n = 1}^{\infty} E_{n-1}\frac{x^{n-1}}{(n-1)!}+ \sum_{n = 1}^{\infty}(-1)^n\frac{x^n}{n!} \text{ Simplifying the sum}\\
	 f(x) - 1 &= xf(x)+ (e^-x -1)  \text { Substituting concise EGF}\\
	 f(x) &= \frac{1}{(1-x)} e^-x\\
	 E_n &=  \sum_{i = 0}^{n} {n \choose i} (-1)^i (n-i)! \text{ By convolution of EGF}\\
	 E_n &= n! \sum_{i = 0}^{n}\frac{-1^i}{i!}
	\end{align*}
	Thus the closed form for the given recurrence relation is $ E_n = n! \sum_{i = 0}^{n}\frac{-1^i}{i!}$. In homework 6 we showed that the number of derangements of $[n]$ is,
	\begin{equation*}
	D_n = \sum_{k = 0 }^{n} (-1)^{k}(n-k)!{n \choose k}
	\end{equation*}
	It should be clear that when we used convolution on $f(x)$ we got the same thing,
	\begin{equation*}
	 E_n &=  \sum_{i = 0}^{n} {n \choose i} (-1)^i (n-i)! 
	\end{equation*}
	Thus it must also be true that $E_n = D_n$

	
	
	\vspace{1in}


\end{enumerate}
\end{document}
