%%% Preamble starts here.
\documentclass{amsart}
%for the heading
\usepackage{fancyhdr, enumerate}
%for the picture. 
\usepackage{tikz}
%adjust the page width
\usepackage[margin=1in]{geometry}

\linespread{1.1}

%command for double parentheses
\newcommand{\textmultiset}[2]{\bigl(\!{\binom{#1}{#2}}\!\bigr)}
\newcommand{\displaymultiset}[2]{\left(\!{\binom{#1}{#2}}\!\right)}
\newcommand\multiset[2]{\mathchoice{\displaymultiset{#1}{#2}}
                                {\textmultiset{#1}{#2}}
                                {\textmultiset{#1}{#2}}
                                {\textmultiset{#1}{#2}}}

%special commands for number sets
\def\RR{{\mathbb R}}
\def\NN{{\mathbb N}}
\def\ZZ{{\mathbb Z}}
\def\QQ{{\mathbb Q}}
\def\CC{{\mathbb C}}

% header
\lhead{\sc  Combinatorics: Homework 4}
\chead{\sc Stefano Fochesatto } 
\rhead{\today}
\cfoot{}
\pagestyle{fancy}

%%%% Main document starts here.

\begin{document}
\thispagestyle{fancy}

\textbf{Directions:} For all numerical problems, a \emph{complete} solutions involves a calculation that ends in a numerical value and a rationale for that calculation.\\

\begin{enumerate}
%%%first problem
\item Problem 2.1.4 Consider the possible function $f \: : \; [7] \to [9].$\\
\begin{enumerate}
\item How many have $f(3)=8$? How many have $f(3)\not=8$? \\
\textbf{Answer: } The total number of functions from $[7] \to [9]$ is counted by $9^{7}$. To count the number of function from $[7] \to [9]$ such that $f(3)=8$ we know that $3$ in the domain is already mapped so all we have to do is count the rest of the functions, $9^{6} = 531441$. To count the number of functions that don't have $3$ in the domain mapped to $8$ we can just calculate the complement, $9^{7} - 9^{6} = 4251528$\\
\item How many have $f(1)\not=5$ and are one-to-one? \\
\textbf{Answer: } To count the number of one-to-one functions that don't map $1$ to $5$ all we have to do is assign and count the number of boxes for the number one ball and then assign the rest, so $8*8_{6} = 161280$.\\
\item How many have $f(i)$ even for all $i$? \\
\textbf{Answer: } Since there are $4$ even numbers in the set $[9]$ we know that each element in the domain $[7]$ is have $4$ possible elements to map to, thus $4^7 = 16384.$\\
\item How many have $rng(f)=\{5,6\}$?\\
\textbf{Answer: } There are $2^7 - 2 = 126$ functions that have a $rng(f)=\{5,6\}$. $2^7$ because each element in the codomain, ie. 5 and 6 can have up to 7 elements mapped. The  $-2$ is ti remove the two functions where the $rng(f)=\{5\}$ and $rng(f)=\{6\}$.\\
\item How many in which $f^{-1}$ is not a function? \\
\textbf{Answer: } Since there exists no bijective functions from $[7] \to [9]$, every function from $[7] \to [9]$ has an inverse which is not a function, $9^{7} = 4782969$.
\end{enumerate}

\vspace{1in}

%%%%second
\item Find the number of onto functions $[k] \to [4].$\\
\textbf{Answer:} The number of onto function from $[k] \to [4]$ can be calculated by $S(k,4)*4!$ such that $k \geq 4$. \\

\vspace{1in}

%%%%%third
\item Problem 2.1.11 Give a combinatorial proof: For $n \geq 1$ and $k \geq 1,$ $2^{kn} > \text{max}\{n^k,k^n\}.$ (Hint: Compare relations to functions.)\\
\textbf{Proof: }Consider that $2^{kn}$ counts the number of relations from sets $k$ and $n$, also note that the number of functions from one set to another is counted by $n^k$ or $k^n$. We know by how functions and relations are defined that the set of all functions is a subset of the set of all relations. Therefore it must be true that $2^{kn} > \text{max}\{n^k,k^n\}$.

\vspace{1in}

%%%fourth
\item Problem 2.2.4 Give combinatorial or bijective proofs of the following. Part of your job is to determine all values of $n$, $k$, and/or $m$ for which the identities are valid.\\

\begin{enumerate}
\item $3^n = \displaystyle{\sum_{k=0}^n {n \choose k} 2^{n-k}}$ \\

\textbf{Answer:} Consider the total of $n$ - digit ternary numbers, we know that for each digit there is 3 options, so $3^n$. We can partition the set of $n$  - digit ternary numbers by the number of k digits that are $2$s. Consider $n = 3$, then,
\begin{align*}
3^3 &= {3 \choose 0}*2^{3-0} + {3 \choose 1}*2^{3-1} + {3 \choose 2}*2^{3-2} + {3 \choose 3}*2^{3-3}\\
27 &= 1*2^{3} + 3*2^{2} + 3*2^{1} + 1*2^{0}\\
&= 8 + 12 + 6 + 1\\
&= 27
\end{align*}
We can see that the choose statement serves to identify the $k$ $2$s in the ternary number, then the $2^{(n-k)}$ term counts the number of ways to fill the rest of the $n-k$ digits. Note that the last term in the sum will always correspond to the ternary number that contains all $2$. For our proof $n \geq 1$ even though the the equation is still satisfied when $n = 0$ it doesn't make sense to have a zero digit ternary number. \\


\item ${n \choose k}{k \choose j}={n \choose j}{n-j \choose k-j}$\\

\textbf{Answer:} First consider a $k$ size committee of $n$ people, and then of the $k$ people in that committee choose $j$ as department heads. The right side of the equation simply chooses the $n$ department heads first and then fills the rest of the $k$ size committee. \\

\item[(d)] $\multiset{n}{k}=\multiset{k+1}{n-1} $\\

\textbf{Answer:} Simplifying the expression above into a choose statement,

\begin{align*}
 \multiset{n}{k} &= \multiset{k+1}{n-1}\\
{n+k-1}\choose{k} &= {{(k+1)+(n-1)-1}\choose{n-1}}\\
{k+n-1}\choose{k} &= {{k+n-1}\choose{n-1}} 
\end{align*}
Now it is clear that, the expression above simply serves to invert the multi-set formula, for example instead of picking $k$ donuts out of $n$ flavors we can pick $n-1$ donuts out of $k+1$ flavors. Things become even clearer when we reduce to the combination formula and think of a $k+n-1$ length binary digit  
because we can see that when we choose $k$ of those digits (right side of equation) it is the same as choosing $n-1$ of those digits. $n \geq k \geq j$ such that $n\geq 1$.
\end{enumerate}

\vspace{1in}

\item Problem 2.2.5 What does $\displaystyle{ {n-1  \choose k-1 }+ {n-2  \choose k-1 }+ {n-3  \choose k-1 }+ \cdots +{k-1  \choose k-1 }}$ equal? Make a conjecture and then give a combinatorial proof.\\

\textbf{Proof:} Consider Pascal's rule
\begin{equation*}
{n\choose k} = {{n-1}\choose {k-1}}+{{n-1}\choose {k}}
\end{equation*}
Now suppose we iterate Pascal's rule over the last term, thus,
\begin{align*}
{n\choose k} &= {{n-1}\choose {k-1}}+{{n-2}\choose {k-1}}+{{n-2}\choose {k}}\\
&= {{n-1}\choose {k-1}}+{{n-2}\choose {k-1}}+{{n-3}\choose {k-1}}+{{n-3}\choose {k}}\\
\end{align*}
The end sum will be, 
\begin{equation}
{n\choose k} &= {{n-1}\choose {k-1}}+{{n-2}\choose {k-1}}+{{n-3}\choose {k-1}}+{{n-3}\choose {k}}+\text{ ... }+{{k-1}\choose {k-1}}+{{k-1}\choose {k}}\\
\end{equation}
We can see that following the ${{k-1}\choose {k-1}}$ term the rest of the iterations will be zero. For this in terms of a combinatorial proof, let's first consider that ${n\choose k}$ counts the number of subsets with $k$ elements from a set with $n$ elements. We can partition these subsets on whether or not they contain the $n$th element, ${{n-1}\choose {k-1}}$ being the term that contains the $n$th element and ${{n-1}\choose {k}}$ begin the element that doesn't contain the $n$th element. This process is repeated on the second term until there are no more elements left.


\vspace{1in}

\item Problem 2.2.7 Determine the number of solutions to each of the following equations. Assume all $z_i$ are nonnegative integers unless stated otherwise.
\begin{enumerate}
\item $z_1+z_2+z_3+z_4=1$\\
\textbf{Answer:} There exists only 4, integer solutions such that $z_i$ are nonnegative integers.
\item $z_1+z_2+10z_3=8$\\
\textbf{Answer:} There exists only $\multiset{2}{8}$, integer solutions simply because $z_3$ must always be zero, $z_i$ are nonnegative integers.\\ 
\begin{equation*}
\multiset{2}{8} = {{9}\choose{8}} = 9
\end{equation*}\\

\item $z_1+z_2+\cdots +z_{20}=401$ where each $z_i \geq 1$\\
\textbf{Answer:} We can count these integer solutions, given the new inequality by, changing the sum. We can reduce the problem to this,
\begin{equation*}
z_1+z_2+\cdots +z_{20}=381 \text{ where each $z_i \geq 0$}.
\end{equation*}
Then we simply calculate the integer solutions through the multiset formula,
\begin{equation*}
\multiset{20}{381} = {{400}\choose{381}} = 1.464*10^{32}
\end{equation*}


\end{enumerate}
\vspace{1in}

\item Problem 2.3.2 For any integer $n \geq 2,$ how many onto function $[n] \to [n-1]$ are possible? Give an answer that does not involve Stirling numbers.\\
\textbf{Answer:} For a function to be onto, for every element in the codomain there must be an element in the domain that maps onto it. First lets partition the $[n]$ elements into $[n-1]$ blocks. By the PHP there will always be a block size two and the rest of the elements are in their own blocks, we can count this by ${{n}\choose{2}}$. Now that we have $n-1$ block all we have to do is map them to the $n-1$ elements in the codomain. Thus the number of onto function $[n] \to [n-1]$ is ${{n}\choose{2}}*(n-1)!$.




Thus the total number of onto functions from $[n] \to [n-1]$ is ${{n}\choose {2}}*(n-1)!$\\

\item Problem 2.3.7 Find the number of equivalence relations on an $n$-set.\\
\textbf{Answer:} There exists a bijection between equivalence relations on the $n$-set and the number of partitions on that set, when we simply define each partition as an equivalence class. Thus if we count up the possible partitions on an $n$-set then we have counted all the possible equivalence relations. So the number of equivalence relations on an $n$-set is simply $B(n)$.

\end{enumerate}
\end{document}
